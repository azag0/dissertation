\section*{Abstract}

The ubiquitous long-range van der Waals interactions play a central role in nearly all biological and modern synthetic materials.
Yet the most widely used theoretical method for calculating material properties, the density functional theory (DFT) in semilocal approximation, largely neglects these interactions, which motivated the development of many different vdW models that can be coupled with DFT calculations.
Despite these efforts, existing vdW models are either limited in scope (atomic models), in efficiency by working with unoccupied one-particle states (e.g., random-phase approximation), or limited to pairwise approximation (nonlocal density functionals).
The work in this thesis paves way towards a unified vdW model that combines best elements from these different classes of the vdW models.

To this end, we developed a unified theoretical framework based on the range-separated adiabatic-connection fluctuation--dissipation theorem that encompasses most existing vdW models.
We show that the formulations of the theorem in terms of the density response function and the nonlocal polarizability are equivalent, introduce the concept of the semilocal effective polarizability and the corresponding effective dipole operator, and discuss the most popular vdW models in terms of these two quantities.
This unified perspective suggests that a particularly effective combination should be that of the local polarizability functionals and the many-body dispersion (MBD) approach based on quantum harmonic oscillators.

We analyze the MBD correlated wave function on the prototypical case of $\pi$--$\pi$ interactions in supramolecular complexes and find that these interactions are largely driven by delocalized collective charge fluctuations, and that the charge density polarization resulting from these fluctuations is well described by the underlying harmonic oscillator.
This demonstrates the close correspondence between the simple harmonic-oscillator model of the polarization, and the actual density response of the true electrons, further supporting the use of polarizability functionals of the density to parametrize the MBD model Hamiltonian.

To identify a balanced short-range density functional to accompany the long-range vdW model, we present a comprehensive study of the interplay between the short-range and long-range energy contributions in eight semilocal functionals and three vdW models on a wide range of systems.
The binding-energy profiles of many of the DFT+vdW combinations differ both quantitatively and qualitatively, and some of the qualitative differences are independent of the choice of the vdW model, establishing them as intrinsic properties of the respective semilocal functionals.
We identify the PBE functional to have the most consistent effective range across different system types.

Finally, we investigate the performance of the Vydrov--Van Voorhis polarizability functional across the periodic table, identify systematic underestimation of the polarizabilities and vdW $C_6$ coefficients for $s$- and $d$-block elements, and develop an orbital-dependent generalization of this functional to resolve the issue.
We establish the quadrupole polarizabilities calculated from such a polarizability functional as a natural parameter governing the range separation in a combined DFT+vdW model.
Overall, our results provide the theoretical framework and key elements that are necessary for a formulation of a general and accurate vdW model.
