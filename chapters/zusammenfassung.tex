% vim: set spelllang=de:

\begin{otherlanguage}{german}
\section*{Zusammenfassung}

Van der Waals-Wechselwirkungen (vdW) sind allgegenwärtig und spielen eine zentrale Rolle in einer großen Anzahl biologischer und moderner synthetischer Materialien.
Die am weitesten verbreitete theoretische Methode zur Berechnung von Materialeigenschaften, die Dichtefunktionaltheorie (DFT) in semilokaler Näherung, vernachlässigt diese Wechselwirkungen jedoch größtenteils, was zur Entwicklung vieler verschiedener vdW-Modelle führte welche mit DFT-Rechnungen gekoppelt werden können.
Ungeachtet dieser Bemühungen sind bestehende vdW-Modelle limitiert entweder in Hinsicht auf ihren Anwendungsbereich (atomistische Modelle), ihre Effizienz im Umgang mit unbesetzten Einteilchen-Zuständen (z.B.
Random-Phase-Approximation) oder auf Zweiteilchen-Näherungen (nichtlokale Dichtefunktionale).
Die hier vorgestelle Arbeit ebnet den Weg hin zu einem vereinheitlichten vdW-Modell welches die besten Elemente dieser unterschiedlichen Klassen von vdW-Modellen vereint.

Zu diesem Zweck haben wir einen vereinheitlichten theoretischen Rahmen geschaffen, der auf  dem Reichweite-separierten Adiabatischer-Zusammenhang-Fluktuations-Dissipations-Theorem aufbaut und die meisten existierenden vdW-Modelle umfasst.
Wir zeigen, dass die Formulierungen des Theorems im Rahmen der Dichte-Antwortfunktion und der nichtlokalen Polarisierbarkeit äquivalent sind, führen das Konzept der semilokalen effektiven Polarisierbarkeit sowie des entsprechenden effektiven Dipoloperators ein und diskutieren die populärsten vdW-Modelle im Kontext dieser beiden Größen.
Diese vereinheitlichte Perspektive legt nahe, dass eine besonders effektive Kombination durch die des Funktionals der lokalen Polarisierbarkeit und des Ansatzes der `Many-Body Dispersion' (MBD), der auf quantenmechanischen harmonischen Oszillatoren beruht, gegeben sein 
sollte.

Wir analysieren die MBD-korrelierte Wellenfunktion am prototypischen Beispiel von $\pi$--$\pi$-Wechselwirkungen in supramolekularen Komplexen und stellen fest, dass diese Wechselwirkungen größtenteils durch delokalisierte kollektive Ladungsfluktuationen entstehen und die aus diesen Fluktuationen resultierende Polarisation der Ladungsdichte gut im Modell des harmonischen Oszillators beschrieben werden kann.
Dies verdeutlicht den engen Zusammenhang zwischen dem einfachen harmonischen Oszillator-Modell für die Polarisation und der tatsächlichen Dichteantwort der wahren Elektronen.
Dies wiederum spricht für die Verwendung von Polarisierbarkeitsfunktionalen der Dichte, um den MBD-Modellhamiltonoperator zu parametrisieren.

Um zu dem langreichweitigen vdW-Modell ein ausgewogenes kurzreichweitiges Dichtefunktional zu identifizieren, präsentieren wir eine umfassende Untersuchung zum Zusammenspiel der kurz- und langreichweitigen Energiebeiträge in acht semilokalen Funktionalen und drei vdW-Modellen für eine große Spanne von Systemen.
Die Bindungsenergieprofile vieler der DFT+vdW-Kombinationen unterscheiden sich sowohl quantitativ als auch qualitativ stark voneinander, wobei einige der qualitativen Unterschiede unabhängig vom vdW-Model sind und damit intrinsische Eigenschaften des verwendeten semilokalen Funktionals darstellen.
Das PBE-Funktional stellt sich als jenes mit dem konsistentesten effektiven Bereich für verschiedene Systemtypen heraus.

Schließlich untersuchen wir die Performance des Vydrov-Van Voorhis-Polarisierbar\-keits\-funktionals über das Periodensystem der Elemente hinweg und identifizieren eine systematische Unterschätzung der Polarisierbarkeiten und vdW-$C_6$-Koeffizienten für $s$- und $d$-Block-Elemente.
Als Lösung entwickeln wir eine orbitalabhängige Verallgemeinerung des Funktionals.
Die aus einem solchen Polarisierbarkeitsfunktional berechneten Quadrupol-Polarisierbarkeiten werden als natürliche Parameter etabliert, die die Bereichsseparierung in einem kombinierten DFT+vdW-Modell regeln.
Insgesamt liefern unsere Ergebnisse den theoretischen Rahmen und die Schlüsselelemente, die für die Formulierung eines allgemeinen und akkuraten vdW-Modell nötig sind.
\end{otherlanguage}
