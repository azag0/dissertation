\prechapter{Preface}

In this thesis, I use the term `vdW force' or `vdW interaction' (rather than dispersion) for the force caused by instantaneous long-range correlation between the motions of electrons, regardless whether in the normal or retarded regime, and `noncovalent interactions' for all intermolecular forces, which \emph{include} the vdW force.

Chapter~\ref{chap:intro} gives a broad nonmathematical background that establishes what van der Waals forces are, the historical development of their understanding, and how they fit within the more general and fundamental physical laws of our world.
Chapter~\ref{chap:electrons} reviews some basic concepts of quantum chemistry and solid-state physics used throughout the thesis.
Chapter~\ref{chap:vdw-methods} develops a formal mathematical classification of existing vdW models, which puts them within a single framework and makes relationships between different models apparent.
Chapter~\ref{chap:mbd} then presents several new developments within one recent successful vdW model, while Chapter~\ref{chap:pi-pi} applies some of them to the problem of $\uppi$--$\uppi$ interactions.
Chapter~\ref{chap:casimir} deals with unification of existing models for microscopic normal and macroscopic retarded vdW forces that provides a single framework for modeling vdW interactions across many orders of magnitude in scale.
Chapter~\ref{chap:xc-functionals} is concerned with the problem of balancing semilocal and nonlocal contributions to the electron correlation energy.
Chapter~\ref{chap:polarizability} presents a new orbital-dependent polarizability functional and outlines its use in the many-body dispersion model.

\begin{flushright}
Jan Hermann
\end{flushright}
