\prechapter{Preface}

This doctoral thesis is a result of my four years at the Fritz Haber Institute in Berlin, which were directed at development of a more general and accurate model of van der Waals (vdW) interactions in molecules and materials.
The presented work contributes to this goal in several ways.
Chapter~\ref{chap:intro} gives a broad conceptual background that establishes what vdW forces are, the historical development of their understanding, and how they fit within the more general and fundamental physical laws of our world.
Chapter~\ref{chap:electrons} reviews basic concepts of quantum chemistry and solid-state physics used throughout the thesis.
Chapter~\ref{chap:vdw-methods} develops a formal mathematical classification of existing methods for modeling vdW interactions, which puts them within a single framework formulated in terms of the nonlocal dipole polarizability, and makes relationships between the different models apparent.
For instance, it shows that the properties of the quantum harmonic oscillator underlie many seemingly unrelated polarizability models, from continuous to coarse-grained functionals of the electron density.
Chapter~\ref{chap:mbd} then presents several new developments within a particular vdW model, the many-body dispersion (MBD) method, while Chapter~\ref{chap:pi-pi} applies the newly derived results for the interacting MBD wave functions to the problem of $\uppi$--$\uppi$ interactions.
This study also demonstrates that the harmonic-oscillator model is able to capture not only the coarse-grained electronic-response propeties in molecules and materials, but also the redistribution of the electron density caused by vdW interactions.
This motivates the focus on the spatial distribution of the polarizability model in the last chapter.
Chapter~\ref{chap:xc-functionals} is concerned with the problem of balancing semilocal and nonlocal contributions to the electron correlation energy, which is central to description of vdW-bound systems in equilibrium.
This work partially rationalizes the empiricism involved in development of new vdW models stemming from the use of damping functions.
Chapter~\ref{chap:casimir} briefly shows that any polarizability model within the MBD framework can be used not only in standard nanoscale vdW models, which assume that the electromagnetic force acts instantly, but also to model microscale systems, where the finite speed of light must be taken into account.
Finally, Chapter~\ref{chap:polarizability} presents a new orbital-dependent polarizability functional of the electron density, and outlines how it can be used within the MBD framework to formulate a new model of vdW interactions.

This thesis would never come to life without the support and advise of my supervisor Alexandre Tkatchenko.
The countless discussions with him inspired many thoughts presented on the following pages.
He also taught me many valuable lessons about scientific writing, publishing, and general wisdom about how modern scientific reasearch is done.
I am indebted to Robert DiStasio for stimulating covnersations and tireless comments about my English---he showed me the art of constructing precise sentences.
My pursue of the doctoral degree would not be possible without the financial support of the Max Plack Society, which was granted by Matthias Scheffler, the director of the theory department at the Fritz Haber Institut in Berlin.
Finally, I would like to express gratitude to all the members of the department, whose excitement about science gave me a great sense of motivation.

\begin{flushright}
Jan Hermann \\
October 2017
\end{flushright}
